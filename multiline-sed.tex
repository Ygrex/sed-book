\chapter{Multiline Sed}
Multilining refers to the state when Sed script operates on the pattern space
where more than one line reside.
\section{Block-wise Operating}
One way of thinking of that is to operate on input block-wise instead of line-wise.
An example would be the problem to filter objects by certain property
where each object described in multiple consequent lines:
\begin{lstlisting}
Country: Washington
City: United States
Population: 705749
====
Country: Moscow
City: Russia
Population: 12506468
====
Country: Mexico City
City: Mexico
Population: 8918653
====
\end{lstlisting}

Here objects are cities with three properties each.
\lstinline{Population} is the property for filtering.
Suppose the filtering condition is at least 10M population.
The expected result is a list of matching objects (only Moscow for this input).

The way to approach the problem presented here is as follows.
Read in the whole block, check the property (if population is
at least eight digits long),
print out matching blocks or discard otherwise.

With the assumption that each block terminates with \lstinline{====} line,
reading the first one can be done by this script:
\begin{lstlisting}
:start
/^(.*\n)?====$/ !{N ; b start}
p ; Q
\end{lstlisting}

The first incantation matches multiline content with \lstinline{====} last line.
As long as it does not hold (note the negation mark), it reads in one more line
from the input and checks against the pattern again. As soon as it matches,
the read block is printed out and execution terminates:
\begin{lstlisting}
$ sed -Enf example.sed < example
Country: Washington
City: United States
Population: 705749
====
\end{lstlisting}

Instead of unconditional printing \lstinline{p} let's add a pattern
matching 10M+ population:
\begin{lstlisting}
s/^Population: [0-9]{8}/&/mp
\end{lstlisting}

Note multiline modifier \lstinline{m} here, which makes carret
\lstinline{^} to match beginning of any line in the pattern space, not
only the first one.
Also quitting with \lstinline{Q} is no longer needed, so that
execution will proceed to the next cycle
and the next block processed:
\begin{lstlisting}
:start
/^(.*\n)?====$/ !{N ; b start}
s/^Population: [0-9]{8}/&/mp
\end{lstlisting}

It now gives the required result:
\begin{lstlisting}
$ sed -Enf example.sed < example
Country: Moscow
City: Russia
Population: 12506468
====
\end{lstlisting}
\section{Line-wise Operating}
In the title of this section "line-wise" refers to the mode of reading the input.
The previous section demontrates, how to read in multi-line block. That lead to
a multi-line content in the pattern space. In this section it will be shown how
multilining can be useful with a line-wise input reading.

In this case multilining appears after transformations of a single line of input.
At some point there are lines appear, which are records separated by newline.
The reason of choosing this exactly separator is a syntax sugar provided by
substitution command in multiline mode (enabled with \lstinline{m} or
\lstinline{M}). Any other separator would give practically the same power of
expressiveness, but in trade of complicating regular expressions.
