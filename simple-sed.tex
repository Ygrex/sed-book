\chapter{Simple Sed}
\section{What is Sed}
Sed in this book refers to GNU Sed most of the time.
GNU Sed is a POSIX compliant version of Sed with GNU specific extensions
distributed under conditions of GPL.

GNU extensions are used extensively across the book, so that other
versions of Sed unlikely will give similar result.

POSIX.1-2017 includes Sed as a mandatory utility:
\href{https://pubs.opengroup.org/onlinepubs/9699919799/utilities/sed.html}{POSIX.1-2017 Sed}.
GNU Sed is fully described at GNU page: \href{https://www.gnu.org/software/sed/manual/sed.html}{sed, a stream editor}.
Directory of diverse resources available ar SF.NET:
\href{http://sed.sourceforge.net/}{the sed \$HOME}.

Simple and intended way of using Sed is processing text line-by-line
in a command line or in shell scripts. Quick example is extracting top-level
domains from a list of FQDNs:

\begin{lstlisting}
$ printf '%s\n' host.com host.org localhost.com |
	sed 's/^.*\.//'
com
org
com
\end{lstlisting}

Next example features filtering out wrong domain names.
For simplicity wrong domain names are
those words that contain no dot:

\begin{lstlisting}[caption={Simple filtering},label={lst:simple-filtering}]
$ printf '%s\n' not-a-domain host.com host.org localhost.com |
	sed -n 's/^.*\.//p'
com
org
com
\end{lstlisting}

Here \lstinline{not-a-domain} is the wrong domain which gets filtered out.

These two examples are quite simple and probably familiar to the reader,
but here are a few Sed concepts worth highlighting very the more
involved examples are introduced.

\subsection{One-line greediness}

Pattern from the first example has assumption about the greediness:
\lstinline{/^.*\./}

It is important for multi-level domain names, for example \lstinline{www.example.com}.
It contains tow substrings matching the pattern (i.e. a substring spanning from the
beginning of a line till a dot character):
\lstinline{www.} and \lstinline{www.example.},
which one matches from the whole string depends on the greediness.
Sed has only greedy matching, in other words the longest possible
substring selected and filtered out after being substituted with an empty string:
\begin{lstlisting}[caption={Simple greedy matching}, label={lst:simple-greedy}]
$ echo www.example.com | sed 's/^.*\.//'
com
\end{lstlisting}

Let's assume for a moment that a non-greedy matching desired, for the above
examples it would mean matching on the left-most domain name segment.
The common approach for single line patterns is to replace
``everything matcher'' \lstinline{/.*/} with ``matching everything but
a separator'' \lstinline{/[^.]*/}. It might be confusing to see a dot
character in two meanings here: as a literal dot, that separates domain
name segments, and a special pattern matching against any character.
Both meanings must be clearly understood differently as having no common
semantics. The resulting non-greedy pattern now looks like:
\begin{lstlisting}[caption={Simple non-greedy matching}, label={lst:simple-nongreedy}]
$ echo www.example.com | sed 's/^[^.]*\.//'
example.com
\end{lstlisting}

This approach is the most simple one. But for sure it has its own limitations
and not universally applicable. In particular, meaning of \lstinline{[^.]*}
changes in multi-line mode (controlled by \lstinline{m} or \lstinline{M}
modifiers).

Just one more example to achieve a non-greedy matching from the left will start
with a greedy matching from the right:
\begin{lstlisting}
$ echo www.example.com | sed 's/\..*$/\n&/'
www
.example.com
\end{lstlisting}

Here the input string transformed into two lines, first one contains exactly
the same substring as matched above with a non-greedy pattern
\lstinline{/^[^.]*/}. The later is to filter it out:
\begin{lstlisting}
$ echo www.example.com | sed 's/\..*$/\n&/ ; s/^.*\n//'
.example.com
\end{lstlisting}
and to get rid of the leading separator (dot character):
\begin{lstlisting}
$ echo www.example.com | sed 's/\..*$/\n&/ ; s/^.*\n\.//'
example.com
\end{lstlisting}
Now the result is the same as in the simple example above~\ref{lst:simple-nongreedy}.

One more non-greedy example, which will be useful going forward, is to replace
first separator by a special character which is not expected to be used anywhere
else in the input:
\begin{lstlisting}
$ echo www.example.com | sed 's/\./\x01/'                
wwwexample.com
\end{lstlisting}
Here \lstinline{\x01} is a non-printable character, which used for that purpose.
The next is to run exactly the same~\ref{lst:simple-greedy} substitution command
but with this special separator:
\begin{lstlisting}
$ echo www.example.com | sed 's/\./\x01/ ; s/^.*\x01//'
example.com
\end{lstlisting}
Note the result is identical to the previous one~\ref{lst:simple-nongreedy}.

\subsection{Pattern space}

As it happens all the time when working in a world of regular expressions,
there is one more overloaded term. The pattern space is a special text buffer in Sed.
All text transformation can only happen in this buffer.

Rules for the pattern space are:
\begin{itemize}
	\item Every Sed script has an input data. On the first execution cycle
		the pattern space contains first line of the input
		excluding newline (newline means a line separator which is
		parameterized).
\begin{lstlisting}
$ printf '%s\n' first second | sed 's/^first$/&/p ; Q'
first
\end{lstlisting}
	\item Input might have no trailing newline, it is indistinguishable
		from Sed script:
\begin{lstlisting}
$ printf 'abc' | sed -n 's/^// ; l ; Q'    
abc$

$ printf 'abc\n' | sed -n 's/^// ; l ; Q'
abc$
\end{lstlisting}
		with one exception, when input is empty:
\begin{lstlisting}
$ printf '' | sed -n 's/^// ; l ; Q'
\end{lstlisting}
		in this case, the script does not execute at all.
		\item Every subsequent cycle receives next input line
			in the pattern space. Any data from the previous
			cycle left in the pattern space gets replaced
			entirely, if not explicitly instructed otherwise.
		\item If instructed explicitly, data from the pattern space
			might be partially passed down to the next cycle and
			optionally mixed with next input line.
		\item It is also allowed to append next input line into the
			pattern space and proceed execution of the current cycle.
\end{itemize}

Last two points are of limited use for there are conditions depending on
the data itself. Trying to pass down part of the pattern space might
occasionally result in reading input, while reading input might occasionally
interrupt the execution of the whole script. For starts it is reasonable to
focus on first three rules and note that pattern space usually does not
keep data between execution cycles.

That is how data arrives to the pattern space. The next aspect is
sending this data to the output stream. It is not worth mentioning
unless an auto-printing mode is enabled.
Since it is enabled by default and the output stream is the
only valuable result of the Sed script execution most of the time,
it is important to track state of the pattern space at the time
it gets sent to the output stream.
In auto-printing mode it happens at the end of each cycle and
implicitly with some commands (quitting command
\lstinline{q} is one notable example).
Things are more simple if printing pattern space is controlled
explicitly, all but very simple examples in this book will have
the auto-printing off.

One more note to the pattern space and output stream interaction
will be important if one expects Sed to be a tool for binary
data processing (which is not). As stated above, it cannot be
known from inside the script itself, if the input line has
a trailing newline or not. That is true, but output will have
the trailing newline depending on its presence in the input:
\begin{lstlisting}
$ echo 'sed' | sed -n 's/^...$/success/p ; Q'    
success

$ printf 'sed' | sed -n 's/^...$/success/p ; Q' | wc -c
7

$ printf 'sed\n' | sed -n 's/^...$/success/p ; Q' | wc -c
8
\end{lstlisting}
Any output command behaves the same way: \lstinline{p},
\lstinline{P}, \lstinline{w}, \lstinline{W} and
implicit printing.
Unambiguous printing \lstinline{l} ignores the distinction
consistently though, it always appends \lstinline{$} and newline,
but it is hardly useful for anything but debugging.

With that all said about the pattern space and printing, let's have
a closer look to the above filtering example~\ref{lst:simple-filtering}.
First of all, it runs in a mode with auto-printing off.
The result is only assembled by calling printing command
\lstinline{p} explicitly. In this case the command is
called indirectly by substitution \lstinline{s} whenever
it succeeds. On its turn it succeeds whenever the
substitution pattern \lstinline{/^.*\./} matches against
the text in the pattern space. The pattern space in this
particular example on each execution cycle receives one
word from list:
\lstinline{not-a-domain}, \lstinline{host.com},
\lstinline{host.org} or \lstinline{localhost.com}.
Those words, which do not match the substitution pattern,
are skipped by the substitution command and the next
execution cycle starts with the next input word placed
in the pattern space. This is the filtering part.

Those words, which happen to match the substitution pattern,
get transformed (the greedily matching substring removed
by this).
Transformation alters the content of the pattern space,
data gets entirely replaced by the substitution result.
After that, substitution continues to printing because of
\lstinline{p} flag (it is not a command but a flag in this
context) and follows to the next execution cycle then.
