\chapter{From Sed to Ed}
As a historical detour there is a good \href{http://www.columbia.edu/~hauben/book/}{Netizens Netbook} to read.
What is up to Sed from it is that there was a text editor Ed
in a wide spread use in small circles of the time.
Limitations of the editor lead Ken Thompson to abstract out
a global pattern finder named grep.
Forseeing the practical value as well as developing the
idea of abstracting text processing tools further,
McMahon wrote Sed.
One can refer to McMahon's
\href{https://people.eecs.berkeley.edu/~clancy/sp.unix.stuff/sed.pdf}
{SED -- A Non-interactive Text Editor} for motivation behind Sed.

One more time to recap is that Sed implements one function
of Ed, evolves it and brings as a standalone tool.

With the close connection of Sed with Ed, it is natural
to master Sed scripting with power of Ed. As a side note,
this whole book and supplementary materials are written in Ed.
Ed itself and Sed in particular were influenced by other
prominent editor QED, but its availability nowadays makes it
not that attractive choice for the purposes of this book.
Also it happened that QED and Ed are essentially similar, so
that it is not that important which one of two is to choose.
What matters here is to really get oneself through that
interface and immerse in that text processing world full of
abstractions and lacking most of the features all modern
text editors provide, but which are rather distracting in studying
Sed way of thinking.

\section{Introduction to Ed}
As pointed out above, Ed was quite influential, so that a few tools
inheritted its concepts at certain extent. This way Ed is kind of
familiar to nearly every experienced *Nix user even today.
Besides that, there are vi/emacs still alive which also share
some ideas from Ed.
Sam/ACME might also count as contemporary successors.
If anyone interested on the connection between editors,
there are \href{https://www.bell-labs.com/usr/dmr/www/qed.html}
{An incomplete history of the QED Text Editor}
and \href{http://eecs.qmul.ac.uk/~gc/history/}{George Coulouris: Bits of History}
available online.

In other words, Ed is not entirely alien today,
but for sure it is very much excentric to believe a reader ever used it.
A short practical introduction follows.
A conceptual implementation can be found in <<Software Tools>> by
Brian W. Kernighan and P.J.Plauger.
It might be interesting to look through along with this book
because Sed is somewhere inside.
